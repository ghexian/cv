% !TEX TS-program = xelatex
% !TEX encoding = UTF-8 Unicode
% !Mode:: "TeX:UTF-8"

\documentclass{resume}
\usepackage{zh_CN-Adobefonts_external} % Simplified Chinese Support using external fonts (./fonts/zh_CN-Adobe/)
%\usepackage{zh_CN-Adobefonts_internal} % Simplified Chinese Support using system fonts
\usepackage{linespacing_fix} % disable extra space before next section
\usepackage{cite}

\begin{document}
\pagenumbering{gobble} % suppress displaying page number

\name{张中博}

% {E-mail}{mobilephone}{homepage}
% be careful of _ in emaill address
\contactInfo{(+86) 17709453094}{boolzz@163.com}{Unity客户端}{GitHub @ghexian}
% {E-mail}{mobilephone}
% keep the last empty braces!
%\contactInfo{xxx@yuanbin.me}{(+86) 131-221-87xxx}{}

% \section{\faGraduationCap\ 教育背景}
\section{教育背景}
\datedsubsection{\textbf{武汉大学},计算机科学与技术,\textit{工程硕士}}{2016.7 - 2018.7}
\datedsubsection{\textbf{吉林大学},软件工程,\textit{工学学士}}{2012.9 - 2016.7}

% \section{\faCogs\ IT 技能}
\section{技术能力}
% increase linespacing [parsep=0.5ex]
\begin{itemize}[parsep=0.2ex]
  \item 熟悉Unity、git
  \item 熟悉c/cpp/csharp/lua/python
  \item 扎实的计算机基础知识
  \item 良好的自驱力,快速学习力
  \item 良好的英文文献阅读力
\end{itemize}

% \end{itemize}

\section{工作经历}
\datedsubsection{\textbf{字节跳动}, 全明星激斗项目组 Unity客户端}{2018.7-至今}
\begin{itemize}
%   \item 飞猪北京前端团队全面负责各交通线的票务(机票/火车票/汽车票) web 应用与事业群基础架构研发
  \item 重构热更新系统。原热更新系统代码偏面向过程,代码重复度高,不便维护。以面向对象方式重构后,减少了代码冗余性,代码结构更清晰。扩展断点续传、zip解压功能。
  \item 重构游戏状态机系统。原系统用枚举区分状态,在一个状态管理器类内写所有的状态逻辑,导致原系统过于复杂。我使用状态模式重构该系统,拆分各个状态到各自的状态类中,相互之间无耦合关系,将原本2000多行的状态管理机类拆到只有160行,使结构更加清晰,状态职责更明确,系统可维护性、可扩展性更好。
  \item 重构Lua层UI框架。原来的UI框架没有系统概念,以窗口为单位,导致系统开发中会产生许多窗口文件夹,文件结构比较乱,也没有事件消息机制,数据更新与界面显示耦合,灵活性较差。重构后,新的UI框架以系统为单位,统一管理系统内的窗口文件、数据文件,并且引入观察者模式,解耦数据更新与界面显示,使得客户端系统文件结构清晰,代码更加灵活,扩展性更好。
  \item 重构资源管理器。原资源管理器接口定义不清晰,代码结构乱,asset与assetbundle管理不清晰。重构后,资源管理器对外明确接口定义,统一外部资源调用,内部区分asset与assetbundle管理,使系统对资源的管理更加清晰。
  \item 处理刘海屏适配问题。设计方案适配刘海机型。在刘海机型上,默认开启刘海,针对android o系统,通过白名单获取刘海尺寸,针对android p使用原生android代码获得刘海尺寸。Unity2018已加入支持。
  \item 优化原配置存储系统。通过在csharp代码中模拟lua的table、value、string结构,直接读取lua内存的方式实现csharp与lua的静态数据共享,降低mono消耗。
  \item 重构编队系统代码结构。编队系统与其他系统耦合过深,编队系统内具有大量其他系统的if判定与特殊逻辑,导致难以维护和扩展。通过参数定制、消息响应与回调绑定,将编队系统进行抽象,显示与数据分离,抽离系统特殊逻辑,简化编队系统。
  \item 开发维护客户端红点系统。通过将红点系统抽象成一棵多叉树的形式,把每一个红点作为一个树节点,外部系统红点是子系统红点父节点,子节点红点计数计入父节点红点计数。红点系统在Lua层实现,客户端系统开发时可以很方便地建立自己的红点树,并且数据处理逻辑中加入红点数据刷新即可实现红点逻辑。且红点数据与显示分离,支持没有显示表现的红点。红点表现支持红点模板,方便统一替换,具有灵活性。
  \item 开发维护pvp、pve系统。
  \item 开发维护战斗镜头系统,可以支持自定义语言的镜头控制,具有状态切换,条件判断等特点。
  \item 解决ios、android打包中遇见的问题。
  \item 对接协议加密sdk。
  \item 升级wwise。
  \item 修复ngui部分bug。
\end{itemize}

% \begin{onehalfspacing}
% \end{onehalfspacing}

% \section{\faInfo\ 个人总结}
\section{个人总结}
% increase linespacing [parsep=0.5ex]
\begin{itemize}[parsep=0.2ex]
  \item 喜欢刷算法 leetcode
  \item 热爱 console game 以及 csgo
  \item 习惯阅读金融刊物、栏目
  \item 追求 work life balance
\end{itemize}

%% Reference
%\newpage
%\bibliographystyle{IEEETran}
%\bibliography{mycite}
\end{document}
