 % !TEX TS-program = xelatex
% !TEX encoding = UTF-8 Unicode
% !Mode:: "TeX:UTF-8"

\documentclass{resume}
\usepackage{zh_CN-Adobefonts_external} % Simplified Chinese Support using external fonts (./fonts/zh_CN-Adobe/)
%\usepackage{zh_CN-Adobefonts_internal} % Simplified Chinese Support using system fonts
\usepackage{linespacing_fix} % disable extra space before next section
\usepackage{cite}

\begin{document}
\pagenumbering{gobble} % suppress displaying page number

\name{张中博}

% {E-mail}{mobilephone}{homepage}
% be careful of _ in emaill address
\contactInfo{(+86) 17709453094}{boolzz@163.com}{UE4客户端}{GitHub @ghexian}
% {E-mail}{mobilephone}
% keep the last empty braces!
%\contactInfo{xxx@yuanbin.me}{(+86) 131-221-87xxx}{}

% \section{\faGraduationCap\ 教育背景}
\section{教育背景}
\datedsubsection{\textbf{武汉大学},计算机科学与技术,\textit{工程硕士}}{2016.7 - 2018.7}
\datedsubsection{\textbf{吉林大学},软件工程,\textit{工学学士}}{2012.9 - 2016.7}

% \section{\faCogs\ IT 技能}
\section{技术能力}
% increase linespacing [parsep=0.5ex]
\begin{itemize}[parsep=0.2ex]
  \item 熟悉UE4、Unity、git
  \item 熟悉c/cpp/csharp/lua
  \item 熟悉dx12,了解基本图形算法
  \item 熟悉数据结构、操作系统、计算机网络、设计模式
  \item 良好自驱学习力,注重团队合作
  \item 良好的英文阅读与听力
\end{itemize}

% \end{itemize}

\section{工作经历}
\datedsubsection{\textbf{腾讯(上海)}, 天美z1工作室内容一组}{2021.5-至今}
\begin{itemize}
  \item 项目Lua框架搭建
  \item 基于umg、unlua实现的UI系统
  \item 基于ECA实现的Gameplay框架
  \item 实现multiplayer相关的玩法逻辑
  \item 接入ds管理sdk
\end{itemize}
\datedsubsection{\textbf{字节跳动(上海)}, 朝夕光年101工作室全明星激斗项目组}{2018.7-2021.4}
\begin{itemize}
  \item 重构Lua层UI框架,以系统为单位,统一管理系统内的窗口文件、数据文件,且引入观察者模式,解耦数据更新与界面显示。
  \item 编辑器扩展,UI窗口生成工具。
  \item 开发维护UI插件,定制化slider。
  \item 重构资源管理器。原资源管理器接口定义不清晰,代码结构乱,asset与assetbundle管理不清晰。重构后,资源管理器对外明确接口定义,统一外部资源调用,内部区分asset与assetbundle管理,使系统对资源的管理更加清晰。
  \item 优化原配置存储系统。通过在csharp代码中模拟lua的table、value、string结构,直接读取lua内存的方式实现csharp与lua的静态数据共享,降低mono消耗。
  \item python实现的xls转lua工具,附带校验规则。
  \item 开发维护编队系统代码,通过参数定制、消息响应与回调绑定,方便其他系统接入。
  \item 开发维护客户端红点系统。通过将红点系统抽象成一棵多叉树的形式,把每一个红点作为一个树节点,外部系统红点是子系统红点父节点,子节点红点计数计入父节点红点计数。
  \item 开发维护部分pvp、pve系统。
  \item 开发维护战斗镜头系统,可以支持自定义语言的镜头控制,具有状态切换,条件判断等特点。
  \item 客户端性能优化,profiler,uwa等。
\end{itemize}

% \begin{onehalfspacing}
% \end{onehalfspacing}

%% Reference
%\newpage
%\bibliographystyle{IEEETran}
%\bibliography{mycite}
\end{document}
